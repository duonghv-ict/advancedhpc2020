\documentclass{article}
\usepackage[utf8]{inputenc}

\title{Report.3.cuda.tex}
\author{havienduong }
\date{November 2020}

\begin{document}

\maketitle

\section{Explain how you implement the labwork}
This labwork is implemented step by step as below:
\begin{verbatim}
    void Labwork::labwork3_GPU() {
    // Calculate number of pixels
    int pixelCount = inputImage->width * inputImage->height;
    outputImage = static_cast<char *>(malloc(pixelCount * 3));

    // Allocate CUDA memory
    uchar3 *devInput;
    uchar3 *devGray;

    cudaMalloc(&devInput, pixelCount * sizeof(uchar3));
    cudaMalloc(&devGray, pixelCount * sizeof(uchar3));    

    // Copy CUDA Memory from CPU to GPU
    char *hostInput = (char *) malloc(pixelCount * 3);
    omp_set_num_threads(ACTIVE_THREADS);

    #pragma omp parallel for
    for (int j = 0; j < 100; j++) {     // let's do it 100 times, otherwise it's too fast!
        for (int i = 0; i < pixelCount; i++) {
            hostInput[i * 3] = (char) (((int) inputImage->buffer[i * 3] + (int) inputImage->buffer[i * 3 + 1] +
                                          (int) inputImage->buffer[i * 3 + 2]) / 3);
            hostInput[i * 3 + 1] = hostInput[i * 3];
            hostInput[i * 3 + 2] = hostInput[i * 3];
        }
    }

    cudaMemcpy(devInput, hostInput, pixelCount * sizeof(uchar3), cudaMemcpyHostToDevice);

    // Processing
    int blockSize = 64;
    int numBlock = pixelCount / blockSize;
    grayscale<<<numBlock, blockSize>>>(devInput, devGray);

    // Copy CUDA Memory from GPU to CPU
    cudaMemcpy(outputImage, devGray, pixelCount * sizeof(uchar3), cudaMemcpyDeviceToHost);

    // Cleaning
    cudaFree(devInput);
    cudaFree(devGray);
}
\end{verbatim}

\section{What’s the speedup?}
When block size is increased, the ellapsed time is faster

\section{Try experimenting with different block size values}
With the block size is 64, we have ellapsed time is 16.3ms:
\begin{verbatim}
    student5@ictserver2:/storage/student5/advancedhpc2020/labwork/build$ ./labwork 3 ../data/cloud.jpeg 
    USTH ICT Master 2018, Advanced Programming for HPC.
    Warming up...
    Starting labwork 3
    labwork 3 ellapsed 16.3ms
\end{verbatim}

\noindent With the block size is 32, we have ellapsed time is 13.0ms:
\begin{verbatim}
    student5@ictserver2:/storage/student5/advancedhpc2020/labwork/build$ ./labwork 3 ../data/cloud.jpeg 
    USTH ICT Master 2018, Advanced Programming for HPC.
    Warming up...
    Starting labwork 3
    labwork 3 ellapsed 13.0ms
\end{verbatim}

\noindent With the block size is 16, we have ellapsed time is 119.3ms:
\begin{verbatim}
    student5@ictserver2:/storage/student5/advancedhpc2020/labwork/build$ ./labwork 3 ../data/cloud.jpeg 
    USTH ICT Master 2018, Advanced Programming for HPC.
    Warming up...
    Starting labwork 3
    labwork 3 ellapsed 119.3ms
\end{verbatim}

\noindent With the block size is 8, we have ellapsed time is 116.7ms:
\begin{verbatim}
    student5@ictserver2:/storage/student5/advancedhpc2020/labwork/build$ ./labwork 3 ../data/cloud.jpeg 
    USTH ICT Master 2018, Advanced Programming for HPC.
    Warming up...
    Starting labwork 3
    labwork 3 ellapsed 116.7ms
\end{verbatim}

\section{Plot a graph of block size vs speedup}
With data set

\end{document}
